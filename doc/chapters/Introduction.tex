\chapter{Introduction}

\section{Motivation and background}

Collecting new data has been increasing more rapidly than algorithms and
computer processing power. The average size of each dataset has also
been increasing. This suggests that the only way to keep up with
analysis is to parallelize algorithms.

One of main drivers of such large datasets is analysis
of genomic and proteomic sequences. Regularities in such data can 
give new insights into how these patterns form and how 
they are related to the other features of the data.

More about why it's important... 

In this thesis we explore an algorithm for finding patterns and show how
abstractions can make it scalable and flexible, and simpler both in 
theory and implementation compared with non-abstract version.

\section{Pattern Discovery}

Pattern discovery is a research area aiming to discover unknown patterns
in a given set of data structures that are frequent and interesting according 
to some measure.

Since the discovery algorithms are highly dependent on the
data structures, that are being searched, the algorithm must be minimal
in the requirements on the dataset to be applicable as wide range as possible.
This also means that the patterns found must be dependent on the initial data.

\section{Structure of the thesis}

First we introduce definitions of our data and patterns in Chapter 2. In
Chapter 3 we describe the general SPEXS algorithm as specified in Vilo et al.
In Chapter 4 we generalize and abstrify the algorithm to get a more flexible
and parallel algorithm. We will describe the implementation consideration of
flexible algorithms in Chapter 5 using the abstract SPEXS algorithm as an
example. Chapter 6 uses the implementation to show how it can be applied.